
\chapter{Thesis Requirements of UCAS}
学位论文是研究生科研工作成果的集中体现,是评判学位申请者学术水平、授予其学位的主要依据,是科研领域重要的文献资料。根据《科学技术报告、学位论文和学术论文的编写格式》(GB/T 7713-1987)、《学位论文编写规则》(GB/T 7713.1-2006)和《文后参考文献著录规则》(GB7714—87)等国家有关标准,结合中国科学院大学(以下简称“国科大”)的实际情况,特制订本规定。

\section{General Requirements}

学位论文必须是一篇(或由一组论文组成的一篇)系统的、完整的学术论文。学位论文应是学位申请者本人在导师的指导下独立完成的研究成果,除论文中已经注明引用的内容外,不得抄袭和剽窃他人成果。对学位论文研究做出重要贡献的个人和集体,均应在文中以明确方式标明。学位论文的学术观点必须明确,且立论正确,推理严谨,数据可靠,层次分明,文字正确、语言通畅,表述清晰,图、表、公式、单位等符合规范要求。

\section{About Your Thesis}

硕士学位论文要选择在基础学科或应用学科中有价值的课题,对所研究的课题有新的见解,并能表明作者在本门学科上掌握了坚实的基础理论和系统的专门知识,具有从事科学研究工作或独立担负专门技术工作的能力。

博士学位论文要选择在国际上属于学科前沿的课题或对国家经济建设和社会发展有重要意义的课题,要突出论文在科学和专门技术上的创新性和先进性,并能表明作者在本门学科领域掌握了坚实宽广的基础理论和系统深入的专门知识,具有独立从事科学研究工作的能力。

\section{Language}

除外国来华留学生及外语专业研究生外,研究生学位论文一般应采用国家正式公布实施的简化汉字撰写;应采用国家法定的计量单位。学位论文中采用的术语、符号、代号在全文中必须统一,并符合规范化的要求。

外国来华留学生可用中文或英文撰写学位论文,但须采用中文封面,且应有详细的中文摘要。外语专业的学位论文等应用所学专业相应的语言撰写,摘要应使用中文和所学专业相应的语言对照撰写。

为了便于国际合作与交流,学位论文亦可有英文或其它文字的副本。

\section{Components}

学位论文一般由以下几个部分组成:中文封面、英文封面、致谢、中文摘要、英文摘要(Abstract)、目录、正文、参考文献、附录、作者简历及攻读学位期间发表的学术论文与研究成果。

\begin{enumerate}
  \item 学位论文题目应当简明扼要地概括和反映出论文的核心内容,一般不宜超过25个汉字(符),英文题目一般不应超过150个字母,必要时可加副标题。

  \item 论文摘要包括中文摘要和英文摘要(Abstract)两部分。论文摘要应概括地反映出本论文的主要内容,主要说明本论文的研究目的、内容、方法、成果和结论。要突出本论文的创造性成果或新见解,不宜使用公式、图表,不标注引用文献。英文摘要(Abstract)应与中文摘要内容相对应。摘要最后另起一行,注明本文的关键词(3-5个),关键词是为了文献标引工作从论文中选取出来,用以表示全文主题内容信息的单词或术语。

  \item 正文是学位论文的主体,包括引言(或绪论)、论文主体及结论等部分。
    \begin{itemize}
      \item 引言(或绪论)应包括选题的背景和意义,国内外相关研究成果述评,本论文所要解决的问题、所运用的主要理论和方法、基本思路和论文结构等。引言应独立成章,用足够的文字叙述,不与摘要雷同。

      \item 论文主体由于涉及不同的学科,在选题、研究方法、结果表达方式等有很大的差异,不作统一的规定。但必须严格遵循本学科国际通行的学术规范,言之成理,论据可靠,实事求是,合乎逻辑,层次分明,简练可读。

      \item 结论是对整个论文主要成果的总结,应明确、精炼、完整、准确。结论应明确指出本研究的创新点,对论文的学术价值和应用价值等加以预测和评价,说明研究中尚难解决的问题,并提出今后进一步在本研究方向进行研究工作的设想或建议。应严格区分本人研究成果与他人科研成果的界限。
    \end{itemize}

  \item 参考文献应本着严谨求实的科学态度,凡学位论文中有引用或参考、借用他人成果之处,均应按不同学科论文的引用规范,列于文末(通篇正文之后)。需正确区分直接引用和转引并明确加以标注。

  \item 学位论文印刷及装订要求:学位论文用A4标准纸打印、印刷或复印,按顺序装订成册。自中文摘要起双面印刷,之前部分单面印刷。论文必须用线装或热胶装订,不使用钉子装订。学位论文封面采用国科大统一规定的学位论文封面格式(http://www.gucas.ac.cn/site/82?u=60473),封面用纸一般为150克(需保证论文封面印刷质量,字迹清晰、不脱落),博士学位论文封面颜色为红色,硕士学位论文封面颜色为蓝色。

  \item 学位论文的提交、保存与使用:学位申请者需按规定向国科大提交学位论文的印刷本和电子版,印刷本和电子版在内容与形式上应完全一致;国科大有权保存学位论文的印刷本和电子版,并提供目录检索与阅览服务,可以采用影印、缩印、数字化或其它复制手段保存学位论文;研究所、国科大有义务保护论文作者的知识产权。涉密学位论文在解密后,须按此规定执行。

  \item 本规定自印发之日起施行【2013年04月07日】,解释权属于校学位评定委员会,由国科大学位办公室负责解释。原《中国科学院研究生院研究生学位论文撰写规定》(院发学位字〔2012〕31号)同时废止。
\end{enumerate}
